\documentclass[a4paper, 12pt]{report}

\usepackage{array}
\usepackage{textcomp}
\usepackage{geometry}
\usepackage{todonotes}
\usepackage[francais]{babel}
\usepackage[utf8]{inputenc}
\usepackage{makeidx}
\usepackage{mdframed}
\usepackage{listings}
\usepackage{indentfirst}
\usepackage{wrapfig}
\usepackage{titlesec}
\usepackage{tikzducks}
\usepackage{titling}
\usepackage{graphicx}
\usepackage{csquotes}
\usepackage{t1enc}
\usepackage[colorlinks = true,
            linkcolor = black,
            urlcolor  = black,
            citecolor = black,
            anchorcolor = black]{hyperref}
\graphicspath{{Schemas/}}

\newcommand{\changeurlcolor}[1]{\hypersetup{urlcolor=#1}}

\definecolor{light-gray}{gray}{0.95} %the shade of grey that stack exchange uses
\lstset{
  literate=
  {á}{{\'a}}1 {é}{{\'e}}1 {í}{{\'i}}1 {ó}{{\'o}}1 {ú}{{\'u}}1 {Á}{{\'A}}1 {É}{{\'E}}1 {Í}{{\'I}}1 
  {Ó}{{\'O}}1 {Ú}{{\'U}}1 {à}{{\`a}}1 {è}{{\`e}}1 {ì}{{\`i}}1 {ò}{{\`o}}1 {ù}{{\`u}}1 {À}{{\`A}}1 
  {È}{{\'E}}1 {Ì}{{\`I}}1 {Ò}{{\`O}}1 {Ù}{{\`U}}1 {ä}{{\"a}}1 {ë}{{\"e}}1 {ï}{{\"i}}1 {ö}{{\"o}}1 
  {ü}{{\"u}}1 {Ä}{{\"A}}1 {Ë}{{\"E}}1 {Ï}{{\"I}}1 {Ö}{{\"O}}1 {Ü}{{\"U}}1 {â}{{\^a}}1 {ê}{{\^e}}1
  {î}{{\^i}}1 {ô}{{\^o}}1 {û}{{\^u}}1 {Â}{{\^A}}1 {Ê}{{\^E}}1 {Î}{{\^I}}1 {Ô}{{\^O}}1 {Û}{{\^U}}1
  {œ}{{\oe}}1 {Œ}{{\OE}}1 {æ}{{\ae}}1 {Æ}{{\AE}}1 {ß}{{\ss}}1 {ű}{{\H{u}}}1 {Ű}{{\H{U}}}1 {ő}{{\H{o}}}1 
  {Ő}{{\H{O}}}1 {ç}{{\c c}}1 {Ç}{{\c C}}1 {ø}{{\o}}1 {å}{{\r a}}1 {Å}{{\r A}}1 {€}{{\EUR}}1 {£}{{\pounds}}1,
  numbers=left,
  numbersep=10pt,
  showspaces=false,
  showstringspaces=false,
  showtabs=false,
  stepnumber=1,
  stringstyle=\color{gray},
  tabsize=4,
  basicstyle=\small,
  keywordstyle=\bf\color{blue},
  backgroundcolor=\color{light-gray},
  commentstyle=\color{ForestGreen},
  showstringspaces=false
}
\geometry{left=3cm, right=2cm, vmargin=2.5cm}

\begin{document}
\section*{Remerciements}
Nous tenons à remercier toutes les personnes qui nous ont aidés dans ce projet :

\vspace{10pt}
\textbf{M. Bruno Guillon}, notre tuteur de projet qui a toujours été présent pour nous conseiller et nous aider, notemment sur les problèmes de VPN
rencontrés.

\vspace{10pt}
\textbf{L'ensemble de l'équipe du Centre de Ressources Informatiques} qui s'est montrée compréhensive et patiente face à nos problèmes liés à leur
service.

\vspace{10pt}
\textbf{Nos camarades du local Isilab}, qui nous ont aidés au diagnostic du problème de VPN.

\vspace{10pt}
\textbf{Mme Murielle Mouzat}, pour son très utile livret de consignes pour le rapport de projet.
\newpage
\section*{Table des illustrations}
\newpage
\section*{\center{Résumé}}
	\indent
	Le but de ce projet est la création d'une application permettant la création, la modification, le partage ainsi que le stockage à distance de chemins cyclistes. 
	Cette application sera installable sur téléphones portables et tablettes Android possédant un GPS. L'application est développée en Java avec le SDK Android, le serveur de stockage des données est lui aussi en Java.
	Ce dernier se charge de l'interfaçage avec la base de données MySQL.
	\\\par
	Le développement a été réalisé avec les environnements de développement Android Studio, Eclipse et Visual Studio Code sous Linux (Kubuntu 19.10 et Ubuntu Budgie 19.04). L'application a été utilisée et testée sur deux appareils mobiles différents.
	Le premier est un appareil récent de Xiaomi, le Mi 9 SE avec une version d'Android personnalisée.
	Le second, plus vieux, est un Samsung Galaxy A5 (2016) avec la dernière mise à jour du constructeur.
	\\\par
	A ce jour, l'application possède quelques fonctionnalités en mode hors connexion.
	La base de données est fonctionnelle, et le serveur peut intéragir avec elle pour une partie des fonctionnalités prévues.
	\\\\
	Mots-clés : Android, Java, MySQL
\section*{\center{Abstract}}
\indent
  The goal of this project is to create a smartphone application which can create, modify, share or save on the cloud cycle tracks. 
  This application will be installable on all android devices with a GPS. 
  It was developped using the Java language and the Android SDK, the server side was also developped in Java and interacts with the MySQL database.
  \\\par
  The development was done using three integrated development environments : Android Studio, Eclipse and Visual Studio Code all of them running on Linux (Kubuntu 19.10 and Ubuntu Budgie 19.04).
  The application was tested with two smartphones, the first one is a modern Xiaomi MI 9 SE with a custom ROM. The second one is an old Samsung Galaxy A5 (2016) with the last official update.
  \\\par
  At this time, the application can some basic offline features. The database is functional and the server can update the database for some features.
  \\\\
  Keywords : Android, Java, MySQL
\newpage
\section*{Table des matières}
\section*{Introduction}

\newpage
\section{Lexique}

\underline{GPS} : \emph{Global Positioning System}, système de géolocalisation mondiale par satellite.

\vspace{10pt}
\underline{Smartphone} : \emph{Téléphone intelligent}, un téléphone mobile doté de fonctionnalités similaires
à un ordinateur (internet, vidéos, musiques, jeux) et de capteurs (gyroscope, GPS).

\vspace{10pt}
\underline{Application mobile} : Logiciel développé pour des supports semblables à des smartphones.

\vspace{10pt}
\underline{Serveur} : Service décentralisé permettant à des clients l'accès à des services ou des données.

\vspace{10pt}
\underline{Base de données} : Ensemble d'informations structurées accessible via un service.

\vspace{10pt}
\underline{Java} : Langage informatique.

\vspace{10pt}
\underline{Kotlin} : Langage informatique.

\vspace{10pt}
\underline{XML} : Langage informatique de structuration de données par balises.

\vspace{10pt}
\underline{Android} : Système d'exploitation créé par \emph{Google}, le plus souvent déployé sur des smartphones.

\vspace{10pt}
\underline{Processus (informatique)} : Programme en cours d'exécution sur un système.

\vspace{10pt}
\underline{Activité} : Composante métier d'une application android.

\vspace{10pt}
\underline{Fragment} : Activité liée qu'à une partie de l'écran.

\vspace{10pt}
\underline{Google Map} : Carte électronique de \emph{Google}

\vspace{10pt}
\underline{Marker} : Affichage d'une position sur une Google Map.

\vspace{10pt}
\underline{Polyline} : Affichage d'un ensemble de traits sur une Google Map.


\section{Introduction}
L'outil numérique a envahi progressivement toutes les sphères de notre vie quotidienne. Cet outil nous semble de
plus en plus indispensable et c'est tout naturellement que, pratiquant une activité de vélo tout terrain en forêt,
nous avons souhaité developper une application en lien.

En effet, en explorant différents chemins on souhaite pouvoir s'en souvenir et les réemprunter. Le projet serait de
proposer la possibilité de repérer, baliser et répertorier ces chemins, puis de les partager avec une communauté
amateur de cyclisme.

Notre problématique est donc de répondre à ce besoin en prenant en compte les restrictions liées au sport. La
solution doit être portable, utilisable pour un cycliste. Elle doit utiliser la géolocalisation et une base de données.

Le smartphone est un support qui convient parfaitement, muni le plus souvent d'un GPS, d'une connexion internet
et d'une taille permettant d'être transporté en vélo.

Notre projet se déroule dans le cadre d'un projet de deuxième année d'école d'ingénieurs en informatique à ISIMA. C'est
dans ce contexte que nous proposons une solution.

Nous verrons tout d'abord dans ce rapport notre démarche pour analyser la problèmatique, puis ce que nous avons mis
en oeuvre pour développer une solution et enfin quel est le résultat produit et ses possibles améliorations.

\chapter{Contexte}

\section{Analyse du problème}
La problématique de ce projet est comment aider un cycliste à réaliser des trajets qui ne sont pas répertoriés sur des cartes. L'idée est qu'un cycliste, 
en particulier ceux qui pratiquent le vélo tout terrain, n'a pas à sa disposition des cartes précises des chemins possibles. En effet,
lorsqu'on s'éloigne des routes pour utiliser des chemins (à travers des forêts par exemple) les cartes papiers ou électroniques ne sont pas
très exhaustives. La solution serait donc de fournir un outil permettant aux cyclistes de réaliser leur propres trajets dans ces
chemins et de pouvoir les partager à d'autres cyclistes qui souhaiteraient effectuer ces trajets.

\section{Proposition d'une solution}
L'outil qui permettrait de résoudre ce problème serait donc utilisable par un cycliste afin de sauvegarder son trajet. La solution la plus évidente
est d'utiliser un smartphone, qui possède un GPS, une interface et une connexion internet. La réponse au problème serait donc une application
smartphone.

\subsection{Fonctionnalités principales}
\begin{enumerate}
  \item L'objectif principal est de capturer via un GPS les coordonnées du trajet effectué par l'utilisateur afin de tracer celui-ci sur une carte
  électronique.
  \item Il faut également pouvoir sauvegarder ce trajet pour que l'on puisse le consulter ultérieurement et s'en servir pour refaire le trajet.
  Idéalement, cette fonctionnalité ressemblera au fonctionnement d'un GPS classique indiquant à la fois notre position et notre progression dans le tracé.
  \item Enfin, les trajets pourront être partagés et utilisés par d'autres utilisateurs de l'application.
\end{enumerate}

\subsection{Fonctionnalités bonus}
\begin{enumerate}
  \item Les trajets faits ne sont pas forcément ce qu'on aurait voulu tracer et sauvegarder dans l'application. Une solution pourrait être de
  rendre possible la modification d'un trajet afin de le redessiner.
  \item Pour rendre l'application plus attractive et moins limitée aux amateurs de cyclisme, elle pourrait adopter le comportement d'un réseau
  social, avec plus d'interactions entre les utilisateurs (fil d'actualité, messages privés). Elle pourrait également s'adresser à d'autres sports utilisant ce principe (course à pied, moto).
\end{enumerate}
\chapter{Conception et Réalisation}
\section{Choix de conception}

\subsection{Base de données}
Nous avons tout d'abord dû choisir quelle technologie nous allions utiliser pour notre base de données.
Nous en connaissions un nombre suffisant pour ne pas en chercher d'autres. Notre choix s'est finalement porté sur MySQL car il répondait à nos besoins, et sa mise en place était plus accessible.
\par
\begin{center}
    Technologies de bases de données connues
    \par
    \begin{tabular}{|l|c|c|c|c|}
        \hline
        Nom & Niveau de connaissance & Facilité d'intégration \\
        \hline
        PostGreSQL & Inconnu & Oui\\
        \hline
        MySQL & Bon & Oui \\
        \hline
        Oracle Database & Bon & Oui\\
        \hline
        sqlite3 & Bon & Non\\
        \hline
    \end{tabular}
    \addcontentsline{lot}{table}{Technologies de bases de données connues}
\end{center}


\subsection{Application}
Pour développer la partie client de l'application, le choix s'est porté naturellement sur une application Android puisque nous possèdons des
smartphones sous Android 9 et Android 7. 
\par
Nous avons donc utilisé Android Studio, un environnement de développement intégré conçu pour générer
des applications Androids. Ce logiciel utilise le langage XML pour la partie "statique", visuelle, et nous laisse le choix entre le langage
Kotlin et le langage Java pour la partie exécution de code.

%\section{Description globale du système}

L'écosystème de notre application est composé de trois parties : l'application, le serveur et la base de données.
\par
L'application contient l'acquisition des données ainsi que l'interface utilisateur. Elle communique avec le serveur via un protocole particulier.
Ce dernier se charge d'accéder à la base de données. C'est à dire qu'il y écrit ou lit des informations et les met sous une forme compréhensible par l'application.
\par
L'avantage de cette architecture est qu'elle permet de travailler en parallèle sur les éléments assez facilement. Cela permet également la modification de l'application (exemple : refonte d'interface), sans avoir besoin d'adapter le serveur.
\section{Description détaillée du système}


\section{Le développement de l'application}
\subsection{Les outils}
\subsubsection{Machine virtuelle}
Concernant la partie serveur, nous avons demandé un support pour héberger la base de données. Isima nous a donné en conséquence un machine
virtuelle sur leurs serveurs. Pour travailler dessus, nous utilisons le \emph{ssh}. Le système d'exploitation installé est \emph{CentOS}.
Pour le savoir il faut lire le contenu du fichier \emph{/proc/version} :
\begin{mdframed}[backgroundcolor=light-gray, roundcorner=20pt,
    leftmargin=0, rightmargin=0, 
    innerleftmargin=20, linecolor=darkgray]
    Linux version 3.10.0-1062.4.3.el7.x86\_64 (mockbuild\@kbuilder.bsys.centos.org) (gcc version 4.8.5 20150623 (Red Hat 4.8.5-39) (GCC) ) \#1 SMP Wed Nov 13 23:58:53 UTC 2019
\end{mdframed}

Cela nous permet de savoir comment installer les programmes dont nous auront besoin. En effet pour utiliser et construire notre base de données
et notre serveur il nous faut quelques outils qu'on installe avec le gestionnaire de paquets de CentOS \emph{yum} :
\begin{lstlisting}[language=bash]
sudo yum install java-1.8.0-openjdk
sudo yum install mysql-server
sudo yum install screen
sudo yum install nc
sudo yum install nmap
\end{lstlisting}

\subsubsection{Base de données}
Nous avons tout d'abord dû choisir quelle technologie nous allions utilisé pour notre base de données.
Nous connaissions un nombre suffisant pour ne pas chercher autre chose et juste faire un choix.
\par
\begin{center}
    Technologies que nous connaissions
    \par
    \begin{tabular}{|l|c|c|c|c|}
        \hline
        Nom & Niveau de connaissance & Facilité d'intégration & License & \\
        \hline
        PostGreSQL & Inconnu & Oui & Oui &\\
        \hline
        MySQL & Bon & Oui & Oui & GPLv2\\
        \hline
        Oracle Database & Bon & Oui & Oui & \\
        \hline
        sqlite3 & Bon & Non & Oui & \\
        \hline
    \end{tabular}
\end{center}

Pour installer le \emph{Système de Gestion de Base de Données}. Il faut tout d'abord mettre un mot de passe administrateur
pour s'y connecter.
\begin{lstlisting}[language=bash]
sudo mysqladmin variables -u root -p
\end{lstlisting}
On peut ensuite modifier le contenu du SGBD en utilisant l'interpréteur de commandes mysql :
\begin{lstlisting}[language=bash]
mysql -u root -p
\end{lstlisting}
Il faut enfin créer et sélectionner la base de données que l'on veut utiliser :
\begin{lstlisting}[language=bash]
CREATE DATABASE <nom de la base de données>
USE DATABASE <nom de la base de données>
\end{lstlisting}


\subsubsection{Le serveur Java}

Pour l'exécution du serveur java, on utilisera Screen.
Screen est un logiciel qui permet de créer des terminaux (appelés \emph{sessions}), de s'y déco-\\nnecter puis reconnecter autant que désiré.
Ces terminaux continueront leur exécution en arrière plan. On peut donc utiliser l'entrée et la sortie standard pour communiquer avec le programme.
Pour se connecter à une session ou la créer si elle n'existe pas, on utilise l'option -R.
\begin{lstlisting}[language=bash]
    screen -R <nom_du_terminal>
\end{lstlisting}

\subsubsection{Côté client}
Pour développer la partie client de l'application, le choix s'est porté naturellement sur une application android puisque nous possèdons des
smartphones sous android 9 et android 7. Nous avons donc utilisé android studio, un environnement de développement intégré conçu pour générer
des applications androids. Ce logiciel utilise le langage XML pour la partie "statique", visuelle, et nous laisse le choix entre le langage
Kotlin et le langage Java pour la partie execution de code. Au début du projet nous avions la volonté de profiter de ce projet pour apprendre
le Kotlin, un langage qui semble très intéressant avec son paradigme fonctionnel. Cependant, nous n'avions également que très peu d'expérience
dans le développement mobile qui est aussi très riche avec beaucoup d'aspects et fonctionnements propres à apprendre. Il s'est très vite révélé
qu'il était très difficile d'avancer le projet en apprenant en parallèle le développement mobile et le Kotlin. De plus, lorsqu'un problème
survient, il est beaucoup plus aisé de trouver de la documentation ou de l'aide avec le langage Java puisque le Kotlin est beaucoup plus récent.
Il a donc été d'un commun accord de reprendre le projet avec le langage Java afin de se concentrer sur l'apprentissage du développement mobile.

\subsection{Base de données}
Le développement de la base de données s'est fait rapidement. Nous avons décidé de 
\begin{figure}[ht]
    \label{Schéma de la base de données}
    \centering
    \includegraphics[scale=0.6]{images/bdd.png}
    \caption{Schéma représentant la base de données}
\end{figure}

\subsection{Le serveur}
\subsubsection{Processus}
Le serveur utilise une architecture avec plusieurs processus. Le processus principal attend une connection provenant d'un client (application).
A chaque connexion un nouveau processus est créé pour intéragir avec le client. Ce processus attend donc un message du client, exécute la commande
SQL nécessaire pour obtenir une réponse et enfin, il répond au client en formattant les données. Un processus supplémentaire existe pour pouvoir
travailler directement avec le serveur, sans passer par un client. Pour cela, l'entrée et la sortie standard sont utilisées et il faut donc un accès direct à la machine.
\subsubsection{Classes}
Le processus principal n'utilise qu'une seule classe \emph{Serveur}, tout comme le processus de communication direct \emph{LocalCommand}.
Cependant le processus de communication avec le client est séparé en plusieurs objets : \emph{Client}, \emph{SQLHandler} et \emph{CommunicationHandler}. 
Le premier est le processus en lui même et utilise les deux autres pour effectuer les tâches qui lui sont assignées. L'objet \emph{SQLHandler} accède à la base de données et formatte les réponses.
Enfin le \emph{CommunicationHandler} gère la communication réseau avec le client, il permet la réception et l'envoie de chaîne de caractères.
Enfin deux classes sont utilisées pour facilité le travail de conversion des formats pour les points : \emph{Point3D} et \emph{Point4D} qui correspondent respectivement à une coordonnée dans l'espace : (x, y, z)
et à une coordonnée dans l'espace et le temps : (x, y, z, t).
\subsubsection{Echange type}
TODO : Diagramme de séquence

\subsubsection{Implémentation}
La création du serveur devait à l'origine être rapide car on le considérait comme étant annexe. C'est pourquoi on a choisi l'utilisation d'une
technologie que nous connaissions déjà : \emph{Java}. Ce langage contient dans sa librairie standard tout ce qu'il faut pour développer un serveur.
De plus il existe une librairie java développée par Oracle pour la communication avec les bases de données MySQL.

Le \emph{Java} est un langage compilé particulier. Le code est tout d'abord compilé dans un langage intermédiaire appelé \emph{Bytecode}. 
Puis, il est exécuté dans une machine virtuelle appelée la \emph {Java Virtual Machine} en utilisant une nouvelle étape de compilation \emph {Just In Time}.
\par
\begin{figure}[ht]
    \label{Schéma compilation Java}
    \centering
    \includegraphics[scale=0.5]{images/java_compilation.png}
    \caption{Schéma représentant la compilation en Java}
\end{figure}
Les programmes développés et compilés en Java peuvent être exécuté sur toutes les machines possédant la JVM installée sans avoir aucun changement de code source ou bien de paramêtre de compilation.
\subsubsection{Fonctionnement des processus}
L'organisation des processus en java est particulière. En effet, il n'y a qu'un seul processus lourd (à la différence des forks du langage C).
Ce processus lourd unique est la JVM. Tous les autres processus, y compris le processus principal de notre programme ne sont que des processus légers.
Il y a donc un partage des resources, et l'échange de données est plus simple. Il faut attention faire cependant aux accès simultanés aux ressources.
\subsubsection{Création de processus}
Le processus principal de notre programme est créé automatiquement par la JVM et il exécute le code de la fonction \emph{public static void main(String[] args)} qui est le point d'entrée de notre programme.
Pour construire d'autres processus il y a plusieurs manières. On va ici se concentrer sur la classe \emph{Thread} et l'interface \emph{Runnable}.
\subsubsection{Utilisation de \emph{Thread}}
Un \emph{Thread} est un objet qui exécute du code dans un autre processus. Pour cela, il suffit de créer un Thread et de le lancer en utilisant la méthode \emph{start()}.
\begin{lstlisting}[language=Java]
Thread monThread = new Thread();
monThread.start();
\end{lstlisting}

Pour changer le code exécuté, il suffit de redéfinir la méthode \emph{void run()} de Thread.
\begin{lstlisting}[language=Java]
Thread monThread = new Thread() {
	@Override
	public void run() {
		System.out.println("Nouveau processus");
	}
}
monThread.start();
\end{lstlisting}

\subsubsection{Patron de conception}
TODO : Fabric/Builder sur le serveur pour les clients\\
TODO : Singleton Serveur + (LocalCommand ???)

\subsubsection{Communication réseau}
En Java, on utilise les objets de type \emph{Socket} pour faire de la communication en réseau. C'est objets utilisent le protocole TCP pour communiquer et permettent donc de s'assurer de l'état de la connection.
Un premier socket (\emph{ServerSocket}) permet d'attendre qu'un client se connecte et de créer un \emph{Socket} pour communiquer avec lui.
\begin{lstlisting}[language=Java]
ServerSocket socket = new ServerSocket(PORT);
while (true) {
	Socket clientSocket = socket.accept();
}
\end{lstlisting}

Ensuite pour échanger avec le client, on utilise les flux d'entrée et de sortie fournis par le Socket.
Les \emph{BufferedReader} et \emph{PrintWriter} sont des objets qui permettent de traiter les flux plus simplement, grâce à des chaînes de caractères.

\begin{lstlisting}[language=Java]
InputStream input = socket.getInputStream();
OutputStream output = socket.getOutputSteam();

BufferedReader reader = new BufferedReader(new InputStreamReader(input));
PrintWriter writer = new PrintWriter(output);
\end{lstlisting}

\subsubsection{Protocole de communication}
    \begin{center}
        Liste des messages possibles dans le sens clients $\rightarrow$ serveur
        \par
        \begin{tabular}{|l|l|}
            \hline
            Commande & Utilisation\\
            \hline
            Subscribe:$<id>:<mdp>$ & Permet de s'inscrire\\
            \hline
            Connect:$<id>:<mdp>$ & Permet de se connecter à son compte\\
            \hline
            \hline
            History:$<debut>:<fin>$ & Permet de récupérer $x$ trajets entre début et fin (en id)\\
            \hline
            Projects:$<debut>:<fin>$ & Permet de récupérer les $x$ projets entre début et fin\\
            \hline
            \hline
            NewP:$<nom>:<x+y+z;...>$ & Ajoute un nouveau projet\\
            \hline
            NewJ:$<nom>:<x+y+z+t;...>$ & Ajoute un nouveau trajet\\
            \hline
            EditP:$<id>:<x+y+z;...>$ & Modifie un projet\\
            \hline
        \end{tabular}
    \end{center}
    \begin{center}
        Liste des messages possibles dans le sens serveur $\rightarrow$ client
        \par
        \begin{tabular}{|l|l|}
            \hline
            Commande & Utilisation\\
            \hline
            Subscribed:$<id>$ & Confirme l'inscription \\
            \hline
            Unsubscribed & Erreur lors de l'inscripton \\
            \hline
            Connected:$<id>$ & Valide la connection à son compte\\
            \hline
            Unconnected & Erreur lors de la connection \\
            \hline
            \hline
            Project:$<id>:<nom>:<x+y+z;...;x+y+z>$ & Envoie des informations sur un projet\\
            \hline
            Journey:$<id>:<nom>:<d>:<x+y+z+t...>$ & Envoie des informations sur un projet\\
            \hline
        \end{tabular}
    \end{center}


\subsection{Développement de l'application android}
\subsubsection{Squelette de l'application}
Afin de comprendre les mécanismes du développement mobile, la première phase a été de simplement créer une application très basique, contenant
uniquement les différentes sections qu'on voudrait développer par la suite, sans leur contenu. Il a fallut donc comprendre le système des 
"activités" et des "fragments" qu'utilise android.
\begin{itemize}
  \item Une activité est une composante métier d'une application android et possède une "View" (un partie graphique).
  \item Un fragment s'apparente grandement à une activité. La différence est qu'un fragment est lié à une partie d'écran et non pas à un écran entier. 
\end{itemize}
Ainsi pour créer les différents onglets, on utilise une activité principale qui contient la barre d'outil en haut avec le nom de l'onglet
dans lequel on se trouve et le bouton permettant d'afficher le menu de navigation. Ce menu est également contenu dans l'activité principale.
Chaque élément de ce menu change le fragment situé en dessous de la barre d'outil. On a donc un fragment pour chaque onglet. 
\begin{figure}[ht]
  \label{Activité-Fragment}
  \centering
  \includegraphics[scale=0.13]{images/activity-fragment.png}
  \caption{Schéma de l'imbrication d'un fragment dans l'activité principale}
\end{figure}

\subsubsection{Création d'un trajet}
Une fois plus à l'aise avec android studio, notre objectif premier était de pouvoir créer un simple trajet. Pour ce faire, on a utilisé un
fragment qu'on affiche lorsqu'on clique sur l'onglet "Nouveau trajet". Dans ce fragment, on a placé deux objets : une carte et un bouton.
Android studio met à disposition des éléments complexe déjà faits et nous permet de les utiliser à travers de multiples fonctions. C'est le
cas pour les boutons et la carte électronique qui est une "Google Map". Ces éléments que l'on place statiquement grâce au langage XML sont
ensuite accessibles dans le code Java avec leur identifiant. On récupère donc les objets créés et on les utilise en récuperant leurs informations
(par exemple quand le bouton est cliqué) ou en chageant leur apparence (par exemple en changeant le texte du bouton ou en ajoutant un trait
sur la carte).
La première étape a donc été de prendre en main les fonctionnalités d'une Google Map. Celles qui nous ont servis pour ce projet sont :
\begin{itemize}
  \item Le placement de la caméra (position et zoom)
  \item Les \emph{Markers}, qui permettent de pointer sur une position précises
  \item La \emph{Polyline}, un outil qui permet de dessiner sur la carte avec une suite de positions
\end{itemize}
Le point suivant a été de comprendre le fonctionnement d'android pour obtenir la localisation géographique du téléphone. Le GPS n'est pas le
seul moyen d'obtenir une position. En effet il existe trois façons d'obtenir une localisation :
\begin{enumerate}
  \item le \emph{GPS\_PROVIDER} (Global Positionning System) utilise les satellites
  \item le \emph{NETWORK\_PROVIDER} utilise les wifis et le antennes téléphoniques que détecte le télephone
  \item le \emph{PASSIVE\_PROVIDER} reçoit les positions passivement lorsque d'autres applications en font la demande
\end{enumerate}
Le \emph{NETWORK\_PROVIDER} et le \emph{PASSIVE\_PROVIDER} ne donnent qu'une localisation globale, peu précise, en se servant de wifis
et d'antennes. Dans notre cas, l'utilisateur sera potentiellement éloigné de ce genre d'appareil. De plus, pour tracer un chemin réalisé
en vélo, il faut priviligier la précision des positions. Ainsi nous avons utilisé essentiellement le \emph{GPS\_PROVIDER}.

Ainsi la création d'un trajet se fait de la manière suivante :
\begin{enumerate}
  \item On clique sur l'onglet "Nouveau trajet" qui fait apparaître le fragment contenant une carte Google et un bouton "Nouveau trajet".
  Si l'application est lancée pour la première fois, une fenêtre contextuelle nous indique qu'il faut autoriser l'application à utiliser le GPS.
  Cette fenêtre permet d'ouvrir les paramètres du smartphone.
  \item On clique sur le bouton "Nouveau trajet". A ce moment l'application prends la localisation acutelle comme point de départ et fait
  apparaître un point sur la carte à cet endroit. De plus la carte zoom et ce centre autour de ce point jusqu'à ce qu'il soit au centre
  avec une vision d'une centaine de mètres de rayon autour. Le texte du bouton change et devient "Arrêter le trajet".
  \item On se déplace. Lorsque le GPS détecte un changement de position, l'application créer un nouveau point dans le trajet. Le \emph{Marker}
  se déplace sur cette nouvelle position et un trait se dessine entre le précédent point et le nouveau.
  \item On clique sur le bouton "Arrêter le trajet". Un fenêtre contextuelle permettant de rentrer le nom du trajet apparaît.
  \item On rentre le nom du trajet dans la zone de texte.
  \item On clique sur le bouton "Valider". Le trajet est créé et on peut recommencer.
\end{enumerate}
\vspace{10pt}

Avec cette première version de la création d'un trajet, il a fallu tester pour se rendre compte sur un vrai trajet si l'application était
assez précise. Pour réaliser des tests, le problème est l'utilisation du GPS, puisque contrairement à ce dont on a l'habitude de développer,
ici l'utilisateur bouge pour utiliser l'application. Il devient donc difficile de tester et corriger de manière répétée puisqu'il faut bouger
un minimum pour observer les changement du GPS, puis revenir sur son poste de travail pour corriger le problème éventuel et remettre
l'application sur le téléphone.

Cependant il existe d'autres manières de tester les applications mobiles. Par exemple android studio propose
un smartphone virtuel qu'on créer sur notre poste de travail afin de le manipuler et faire des tests. Pour la partie GPS, on peut injecter
à cette machine virtuelle de fausses coordonnées. Le problème lié à cette méthode est que la virtualisation d'un smartphone coûte très cher
en ressources. Ainsi android studio en plus de la machine virtuelle demandent un ordinateur puissant.

On peut également émuler les coordonnées GPS sur un smartphone réel grâce à des applications (disponibles sur le marché \emph{Play Store}).
On indique quelles doivent êtres nos coordonnées et notre GPS fait comme si il y était. Le problème soulevé par cette émulation, que ce soit
sur une machine virtuelle ou avec une application sur un smartphone, est que ça ne prend pas en compte les erreurs de positionnement GPS.
En effet lors d'un test réel, on a pu observer que le tracé du trajet se fait bien mais que par moment le GPS indique des positions fausses,
ce qui a pour effet de tracer deux traits entre ce point et le chemin qu'on parcourt. Pour pallier ce problème, la solution pourrait être
de capturer la vitesse que nous fournis le GPS et calculer si le nouveau point qu'on veut ajouter au tracé est atteignable avec cette vitesse.
Cependant la vitesse fournis n'est pas fiable lorsqu'on bouge lentement. On a donc décidé de faire ce traitement avec une vitesse maximale,
assez haute pour ne pas être atteinte par un cycliste.
\begin{figure}[ht]
    \label{Différence entre avant et après correction}
    \centering
    \includegraphics[scale=0.6]{images/avant-apres.png}
    \caption{Différence entre un trajet avec le problème d'imprécision du GPS et un trajet avec la correction apportée}
\end{figure}

\subsubsection{Sauvegarde des trajets}
Au début du projet, les trajets effectués par l'utilisateur étaient destinés à être enregistrés dans une base de donnée. Ainsi, les utilisateurs
pouvaient utiliser plusieurs appareils sans perdre leurs trajets et cela aurait permis plus tard le partage de ceux-ci. Cependant, après avoir
rencontrés certains problèmes pour relier l'application smartphone au serveur, il a été décidé de sauvegarder les trajets dans des fichiers,
au moins temporairement.

Por simplifier la communication potentielle avec le serveur, une librairie Java a été réalisé pour représenter un trajet. Cette librairie
contient deux objets essentiels \emph{Journey} et \emph{JourneyHistory}. L'un représentant un trajet et l'autre un ensemble de trajet
(utilisé plus tard pour l'historique). L'objet trajet n'est qu'une abstraction d'un tableau de localisations. Ces localisations possèdent
quatres composantes :
\begin{itemize}
  \item la latitude de la position
  \item la longitude de la position
  \item l'altitude de la position (pour le dénivelé du trajet)
  \item la date à laquelle le point a été pris
\end{itemize}
Ainsi à chaque nouveau point lors d'un trajet, c'est dans cet objet qu'on le stocke. Puis lorsque l'utilisateur finit son trajet,
on l'ajoute à l'objet \emph{JourneyHistory} et on écris dans un fichier le contenu de l'objet (son nom, sa date de création et chaque
point qu'il contient).

\subsubsection{Historique}
L'objet \emph{JourneyHistory} contient en mémoire tous les trajets effectués par l'utilisateur tant que l'application est en fonctionnement.
Ainsi, lorsque l'application se démarre, on charge dans cet objet tous les trajets sauvegardés dans des fichiers. De cette manière, on peut
afficher le contenu de l'onglet "Historique" sans lire des fichiers à chaque fois.

L'affichage de cet onglet se fait via des "MaterialCards". Google met à disposition un certain nombre de composants préfaits. Ceux-ci sont
plus "design" et permettent de garder un visuel cohérent simplement. Ici nous avons utilisé les "Cards" qui sont des conteneurs pour d'autres
objets comme des images ou du texte. On ajoute simplement dans ce conteneur les éléments qui le compose et la mise en forme se fait quasi
automatiquement.

Dans l'historique, chaque "Card" représente un trajet. Une "Card" est composée d'une capture d'écran de la carte prise à la fin du trajet,
du nom du trajet, de sa date de création et de sa durée. Ainsi, lorsqu'on affiche le fragment "Historique", on créer une "Card" pour
chaque trajet dans le \emph{JourneyHistory}. Ceux-ci sont triés par date de création, on les affiche donc du plus récent au plus vieux.

Afin de pouvoir consulter les trajets de manière plus précise, le dernière fonctionnalité ajoutée à été de pouvoir cliquer sur chaque "Card"
de l'historique. Cela a pour effet d'amener vers une autre page, une autre activité permettant de consulter un trajet. Cette page contient
une Google Map avec le trajet dessiné dessus. En dessous on retrouve le nom, la date et la durée du trajet. A l'origine, cette page devait
être plus complète, avec par exemple le dénivelé du trajet, la météo ou des statistiques. De plus la carte affichée n'est ici que pour
permettre à l'utilisateur d'avoir une vue globale du trajet en manipulant la carte. Elle ne remplie pas la fonction voulue à l'origine
qui était de fonctionner comme un GPS de voiture, guidant l'utilisateur au court de son trajet.

\section{Tests effectués}
\subsection{Bibliothèque de gestion des trajets}
Nous avons mis en place lors du développement de la bibliothèque un ensemble de tests unitaires à l'aide de JUnit.
Ces tests nous ont permis de vérifier que les fonctionnalités de la bibliothèque étaient implémentées correctement.
Nous pouvions donc exclure la bibliothèque lorsqu'un problème était rencontré dans l'application.
\subsection{Application}
L'application a été testé comme indiqué précedemment sur nos téléphones privés. Nous avons donc pu utiliser une version d'android récente comme une plus vieille. On a pu également s'assurer que l'interface s'adapte correctement à des écrans de résolutions et formats différents (16:9 et 21:9).
Un des téléphones, le plus récent, a été testé avec des versions d'Android personnalisées. La première \emph{XenonHD} est dérivée d'une version de LineageOS basée sur Android 9. La seconde \emph{crDroid} est dérivée directement de la version de Google d'Android 10. Nous avons donc couvert un panel assez large d'appareil, même si un test avec une tablette aurait pu s'avérer pertinent.
\graphicspath{{Others/}}

\section{Déroulement du projet}
\subsection{Organisation théorique du travail}
\subsubsection{Répartition des tâches et prévision de l'emploi du temps}
Le projet fut dès le départ pensé dans le but d'être simple à séparer sous forme de modules, permettant de travailler en parallèle sur plusieurs fonctionnalités.
Comme nous avions nous-mêmes proposé le sujet, il fut assez difficile de prévoir une charge de travail associée à chaque module. Nous avons donc estimé 
de manière très grossière le temps de travail par module. Pour être sûr de pouvoir ajuster le déroulement du projet, nous avons prévu des modules de durées différentes permettant ainsi
de choisir un module en fonction du temps restant, c'est pour cette raison que la durée estimée est supérieure au 60 heures par personne que nous sommes censés faire.
\par
Nous avions prévu, lors de notre premier rendez-vous avec notre tuteur, de travailler 4 heures par semaine de cours et de ne pas travailler les semaines de vacances.
Nous avons sommes donc parvenu à finaliser l'emploi du temps suivant, qui n'avait pas pour but d'être suivi à la lettre.
\vfill
\begin{figure}[!h]
    \begin{center}
        \includegraphics[height=6cm]{test-1.png}
        \caption{Première partie du diagramme de Gantt prévisionnel}
    \end{center}
\end{figure}
\subsubsection{Explications sur les tâches}
Les deux premiers objectifs fixés étaient assez simples, leur but étaient de nous laisser le temps d'être à l'aise avec les technologies choisies.
Nous devions prévoir la base de données, c'est à dire la concevoir et la mettre en place sur la machine virtuelle. En parallèle de celà, nous devions réussir à récupérer les coordonnées GPS du téléphone, et réussir à les afficher.
\par
Les objectifs suivants étaient d'enrichir l'expérience utilisateur en améliorant l'interface. Nous voulions en premier permettre la gestion d'un compte utilisateur depuis l'application, ce qui implique un écran de connexion, un écran de création de comptes ainsi qu'un écran de gestion de comptes.
Dans un second point (développé en parallèle) nous devions enrichir l'interface fonctionnelle de l'application, c'est à dire insérer une interface contenant une carte sur laquelle notre trajet serait affichable (celà sous entend de stocker les coordonnées aquises).
\par
Nous avions prévu de faire évoluer l'application en rajoutant du contenu. Il aurait fallu ajouter des statistiques plus complètes sur les trajets effectués, comme par exemple$\ :$ le dénivelé, la météo ou bien une estimation des calories dépensées.
Il fallait également introduire une gestion des utilisateurs plus développée, qui permettrait de gérer plus finement les droits d'accès. On aurait ainsi pu dire qu'un autre utilisateur avait participé à un trajet, ou bien qu'il avait le droit d'en modifier le contenu.
De la même manière, on aurait pu créer des groupes d'utilisateurs pour un club par exemple. Dans ces groupes tout le monde aurait accès en lecture uniquement sauf les administrateurs. Ainsi un club sportif aurait pu utiliser l'application pour organiser des séances de randonnées.
\par
Un des derniers points à mettre en place était l'affichage des statistiques précédemments acquises sous la forme de graphique où l'on aurait pu choisir l'échelle, et les trajets qui rentraient en compte.
Le dernier point était radicalement plus difficile à traiter, nous voulions finir le développement de l'application en la faisant se rapprocher d'un réseau social. On aurait alors pu avoir des amis, un fil d'actualité contenant les trajets (publics) de nos amis. On aurait aussi pu partager
nos trajets via des liens webs, qui auraient été ouvrables uniquement par notre application.
\par
Enfin il y avait la dernière tâche qui semble évidente qui était la rédaction du rapport. Nous avions prévu de prendre des notes au fur et à mesure du développement du projet pour parvenir à rédiger le rapport plus efficacement.
\vfill
\begin{figure}[!h]
    \begin{center}
        \includegraphics[height=6cm]{test-2.png}
        \caption{Seconde partie du diagramme de Gantt prévisionnel}
    \end{center}
\end{figure}






\subsection{Organisation réelle du travail}
\subsubsection{Répartion des tâches et emploi du temps}
Au lieu de débuter le projet comme prévu$\ :\ $chacun sur un module, nous avons préféré faire quelques séances de travail en commun afin de découvrir ensemble l'environnement android, et de nous mettre entièrement d'accord sur la suite.
\par
Nous avons ensuite séparé le travail, un s'est chargé du développement android et l'autre de la base de données. Développer sous android implique nécessairement d'en étudier plus le fonctionnement. D'autant plus que nous voulions utiliser Kotlin qui est un langage que nous ne connaissions pas du tout.
Pour la base de données le plus difficile était d'en faire l'installation et la configuration.
\par
Les partiels qui ont suivi nous ont posé beaucoup de problèmes, nous avons été assez surpris par la charge de travail à fournir sur deux semaines. En plus nous avons eu plusieurs travaux pratiques importants à rendre la semaine suivante.
Nous nous sommes replongés dans le travail la semaine précédant les vacances. Nous avons eu le temps d'intégrer l'interface finale de l'application avec le menu sur le côté gauche. Et nous avons commencé le développement du serveur qui allait interagir avec la base de données.
\par
Ensuite il y a eu les vacances et un problème abordé dans la partie "Problèmes rencontrés". Puis nous avons commencé à développer la communication entre le serveur et l'application ainsi que l'affichage d'un trajet sur une carte. Un second problème est survenu ce qui nous a imposé de mettre en pause le développement du serveur, pour finalement ne jamais le reprendre.
Nous avons donc commencé le rapport. L'application s'est étoffée pour fournir l'enregistrement de trajets en local, ainsi que leur sauvegarde (sur le téléphone) sous la forme d'un historique (accessible depuis l'application).
Ces dernières fonctionnalités étant fournies par une bibliothèque de gestion de nos structures de données.
\par
Enfin nous avons dû arrêter le développement de l'application pour nous concentrer sur l'écriture du rapport de projet.
\vfill
\begin{figure}[!h]
    \begin{center}
        \includegraphics[height=8cm]{reel-1.png}
        \caption{Première partie du diagramme de Gantt réel}
    \end{center}
\end{figure}
\newpage
\subsubsection{Méthodologie de travail}
Nous avons essayer autant que possible de nous organiser à l'avance sur le travail. C'est à dire qu'en fin de chaque séance nous avons essayé de fixer un objectif à réaliser à la séance suivante.
Pendant le deuxième semestre l'emploi du temps scolaire que nous avions a fait que nous avons travaillé plus souvent séparément. Nous parlions alors régulièrement de notre avancement respectif en se donnant une date limite pour chaque tâche.
\par
Lors du développement de la bibliothèque de gestion des trajets, nous avons naturellement mis en place une méthode de travail agile, sous la forme d'itérations. Ainsi lorsqu'une fonctionnalité de l'application avait besoin d'un accès à la structure de données, cette partie de la bibliothèque était développée.
Cela nous a permi de travailler efficacement, en mettant le doigt rapidement sur ce qui ne convenait pas à l'application.
\par
Pour ce qui est de la communication avec notre tuteur, nous devions envoyer des mails toutes les semaines pour qu'il puisse suivre l'avancement du projet. Dans la pratique, nous avons été moins assidus. Nous avons cependant gardé le contact au fur et à mesure des évolutions, ainsi que lorsque nous avions des difficultés.
\vfill
\begin{figure}[!h]
    \begin{center}
        \includegraphics[height=8cm]{reel-2.png}
        \caption{Seconde partie du diagramme de Gantt réel}
    \end{center}
\end{figure}
\newpage




\subsection{Problèmes rencontrés}
\subsubsection{Langage Kotlin}
Lors de ce projet nous avons rencontré plusieurs problèmes majeurs qui nous ont obligés à changer l'orientation du projet.
Le premier problème était lié au langage, nous avions prévu de développer l'application en utilisant le langage Kotlin, qui est le nouveau langage officiel.
C'est unn langage qui semble très intéressant avec son paradigme fonctionnel. Cependant, nous n'avions également que très peu d'expérience
dans le développement mobile qui est aussi très riche avec beaucoup d'aspects et fonctionnements propres à apprendre. Il s'est très vite révélé
qu'il était très difficile d'avancer le projet en apprenant en parallèle le développement mobile et le Kotlin. De plus, lorsqu'un problème
survient, il est beaucoup plus aisé de trouver de la documentation ou de l'aide avec le langage Java puisque le Kotlin est beaucoup plus récent.
Il a donc été convenu de reprendre le projet avec le langage Java afin de se concentrer sur l'apprentissage du développement mobile.
\subsubsection{Machine virtuelle}
Nous avons ensuite eu des problèmes liés au matériel qui sont indépendants de notre volonté.
Tout d'abord, pendant les vacances de Noël, nous n'avons pas pu avancer autant que nécessaire, car la machine virtuelle que nous avait fournie l'ISIMA ne fonctionnait pas.
Nous avons donc passé plusieurs jours à essayer de trouver où était notre erreur de configuration, avant de se rendre compte que la machine virtuelle, lors d'une maintenance, avait était éteinte mais pas rallumée.
\subsubsection{Réseau}
Enfin le problème qui nous a le plus bloqué est survenu lorsque nous avons voulu connecter l'application au serveur. En effet, pour accéder à la machine virtuelle il faut obligatoirement être connecté au réseau privé de l'ISIMA.
Pour celà l'ISIMA fournit un fichier de configuration pour le logiciel OpenVPN. Cette configuration fonctionne parfaitement sur ordinateur. Cependant lorsque nous l'avons mis sur nos téléphones, nous perdions tout accès à internet.
Nous avons ensuite contacté le service informatique de l'école, qui a accepté de prendre un rendez-vous pour essayer de diagnostiquer plus précisémment le problème. Nous avons lors de ce rendez-vous, réussi à confirmer que le problème venait
du fichier de configuration qui utilisait un paramètre de compression des données non-compatible avec la version android du logiciel. Nous avons ensuite essayer de trouver des solutions pour pouvoir continuer le projet.
Une fois les délibération terminée, il fut conclu que nous n'aurions pas de solution dans les délais qui nous étaient imposés. Nous avons donc arrêter le développement de la partie réseau et serveur pour nous concenter sur un stockage local des informations.

\chapter{Résultats et discussions}

\newpage
\section{Webographie}
Documentation Java, \changeurlcolor{blue}\href{https://docs.oracle.com/en/java/}{https://docs.oracle.com/en/java/} consultée d'octobre 2019
à février 2020.

\vspace{10pt}
Documentation Android, \changeurlcolor{blue}\href{https://developer.android.com/docs}{https://developer.android.com/docs} consultée d'octobre 2019
à février 2020.

\vspace{10pt}
Material design, \changeurlcolor{blue}\href{https://material.io/}{https://material.io/} consultée en février 2020.

\end{document}

