\documentclass[a4paper, 11pt]{report}

\usepackage{array}
\usepackage{textcomp}
\usepackage{geometry}


\geometry{hmargin=2cm, vmargin=2cm}

\begin{document}

\chapter{Choix BDD}
\section{Base de données}
\subsection{Choix de la base de donnée}
	Nous avons tout d'abord dû choisir quelle technologie nous allions utilisé pour notre base de données.
	Nous connaissions en connaissions un nombre suffisant pour ne pas chercher autre chose et juste faire un choix.
	\par
	\begin{center}
		Technologies que nous connaissions
		\par
		\begin{tabular}{|l|c|c|c|c|}
			\hline
			Nom & Niveau de connaissance & Facilité d'intégration & License \\
			\hline
			PostGreSQL & Inconnu & Oui & Oui\\
			\hline
			MySQL & Bon & Oui & Oui & GPLv2\\
			\hline
			Oracle Database & Bon & Oui & Oui \\
			\hline
			sqlite3 & Bon & Non & Oui \\
			\hline
		\end{tabular}
	\end{center}
\section{Communication avec l'application}
\subsection{Serveur}
\subsection{Protocole}
	\begin{center}
		Liste des messages possibles dans le sens clients $\rightarrow$ serveur
		\par
		\begin{tabular}{|l|l|}
			\hline
			Commande & Utilisation\\
			\hline
			Subscribe:$<id>:<mdp>$ & Permet de s'inscrire\\
			\hline
			Connect:$<id>:<mdp>$ & Permet de se connecter à son compte\\
			\hline
			\hline
			History:$<debut>:<fin>$ & Permet de récupérer $x$ trajets entre début et fin (en id)\\
			\hline
			Projects:$<debut>:<fin>$ & Permet de récupérer les $x$ projets entre début et fin\\
			\hline
			\hline
			NewP:$<nom>:<x+y+z;...>$ & Ajoute un nouveau projet\\
			\hline
			NewJ:$<nom>:<x+y+z+t;...>$ & Ajoute un nouveau trajet\\
			\hline
			EditP:$<id>:<x+y+z;...>$ & Modifie un projet\\
			\hline
		\end{tabular}
	\end{center}
	\begin{center}
		Liste des messages possibles dans le sens serveur $\rightarrow$ client
		\par
		\begin{tabular}{|l|l|}
			\hline
			Commande & Utilisation\\
			\hline
			Subscribed:$<id>$ & Confirme l'inscription \\
			\hline
			Unsubscribed & Erreur lors de l'inscripton \\
			\hline
			Connected:$<id>$ & Valide la connection à son compte\\
			\hline
			Unconnected & Erreur lors de la connection \\
			\hline
			\hline
			Project:$<id>:<nom>:<x+y+z;...;x+y+z>$ & Envoie des informations sur un projet\\
			\hline
			Journey:$<id>:<nom>:<d>:<x+y+z+t...>$ & Envoie des informations sur un projet\\
			\hline
		\end{tabular}
	\end{center}
\chapter{Serveur}
\section{Architecture}
	\subsection{Processus}
		Le serveur utilise une architecture avec plusieurs processus. Le processus principal attend une connection provenant d'un client (application).
		A chaque connection un nouveau processus est créé pour intéragir avec le client. Ce processus attend donc un message du client, exécute la commande
		SQL nécessaire pour obtenir une réponse et enfin, il répond au client en formattant les données. Un processus supplémentaire existe pour pouvoir
		travailler directement avec le serveur, sans passer par un client. Pour cela, l'entrée et la sortie standard sont utilisées et il faut donc un accès direct à la machine.
	\subsection{Classes}
		Le processus principal n'utilise qu'une seule classe \emph{Serveur}, tout comme le processus de communication direct \emph{LocalCommand}.
		Cependant le processus de communication avec le client est séparé en plusieurs objets : \emph{Client}, \emph{SQLHandler} et \emph{CommunicationHandler}. 
		Le premier est le processus en lui même et utilise les deux autres pour effectuer les tâches qui lui sont assignées. L'objet \emph{SQLHandler} accède à la base de données et formatte les réponses.
		Enfin le \emph{CommunicationHandler} gère la communication réseau avec le client, il permet la réception et l'envoie de chaîne de caractères.
		Enfin deux classes sont utilisées pour facilité le travail de conversion des formats pour les points : \emph{Point3D} et \emph{Point4D} qui correspondent respectivement à une coordonnée dans l'espace : (x, y, z)
		et à une coordonnée dans l'espace et le temps : (x, y, z, t).
	\subsection{Echange type}
		TODO : Diagramme de séquence
\section{Implémentation}
	\subsection{Langage}
		TODO : Pourquoi java ?
	\subsection{Fonctionnement des processus}
		TODO : Runnable, Threads
	\subsection{Patron de conception}
		TODO : Fabric/Builder sur le serveur pour les clients\\
		TODO : Singleton Serveur + (LocalCommand ???)
	\subsection{Communication réseau}
		TODO : Expliquer Serveur Socket et Socket, le accept()\\
		TODO : Insérer le protocole de communication écrit plus haut
\end{document}

