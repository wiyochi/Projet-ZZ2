\section{Lexique}

\underline{GPS} : \emph{Global Positioning System}, système de géolocalisation mondiale par satellite.

\vspace{10pt}
\underline{Smartphone} : \emph{Téléphone intelligent}, un téléphone mobile doté de fonctionnalités similaires
à un ordinateur (internet, vidéos, musiques, jeux) et de capteurs (gyroscope, GPS).

\vspace{10pt}
\underline{Application mobile} : Logiciel développé pour des supports semblables à des smartphones.

\vspace{10pt}
\underline{Serveur} : Service décentralisé permettant à des clients l'accès à des services ou des données.

\vspace{10pt}
\underline{Base de données} : Ensemble d'informations structurées accessible via un service.

\vspace{10pt}
\underline{Java} : Langage informatique.

\vspace{10pt}
\underline{Kotlin} : Langage informatique.

\vspace{10pt}
\underline{XML} : Langage informatique de structuration de données par balises.

\vspace{10pt}
\underline{Android} : Système d'exploitation créé par \emph{Google}, le plus souvent déployé sur des smartphones.

\vspace{10pt}
\underline{Processus (informatique)} : Programme en cours d'exécution sur un système.

\vspace{10pt}
\underline{Activité} : Composante métier d'une application android.

\vspace{10pt}
\underline{Fragment} : Activité liée à seulement une partie de l'écran.

\vspace{10pt}
\underline{Google Map} : Carte électronique de \emph{Google}

\vspace{10pt}
\underline{Marker} : Affichage d'une position sur une Google Map.

\vspace{10pt}
\underline{Polyline} : Affichage d'un ensemble de traits sur une Google Map.

