\section{Analyse du problème}
\subsection{Le problème}
La problèmatique de ce projet est comment aider un cycliste à réaliser des trajets qui ne sont pas répertoriés sur des cartes. L'idée est qu'un cycliste, 
en particulier ceux qui pratiquent le vélo tout terrain, n'a pas à sa disposition des cartes précises des chemins possibles. En effet,
lorsqu'on s'éloigne des routes pour utiliser des chemins (à travers des forêts par exemple) les cartes papiers ou électroniques ne sont pas
très exhaustives. La solution serait donc de fournir un outil permettant aux cyclistes de réaliser leur propres trajets dans ces
chemins et de pouvoir les partager à d'autres cyclistes qui souhaiteraient effectuer ces trajets.

\subsection{Proposition d'une solution}
L'outil qui permetterait de résoudre ce problème serait donc utilisable par un cycliste afin de sauvegarder son trajet. La solution la plus évidente
est d'utiliser un smartphone, qui possède un GPS, une interface et une connexion internet. La réponse au problème serait donc une application
smartphone.

\subsubsection{Fonctionnalités principales}
\begin{enumerate}
  \item L'objectif principal est de capturer via un GPS les coordonnées du trajet effectué par l'utilisateur afin de tracer celui-ci sur une carte
  électronique.
  \item Il faut également pouvoir sauvegarder ce trajet pour que l'on puisse le consulter ultérieurement et s'en servir pour refaire le trajet.
  Dans l'idéal cette fonctionnalité ressemblera au fonctionnement d'un GPS classique indiquant à la fois notre position et notre progression dans le tracé.
  \item Enfin, les trajets pourront être partagés et utilisés par d'autres utilisateurs de l'application.
\end{enumerate}

\subsubsection{Fonctionnalités bonus}
\begin{enumerate}
  \item Les trajets faits ne sont pas forcément ce qu'on aurait voulu tracer et sauvegarder dans l'application. Une solution pourrait être de
  rendre possible la modification d'un trajet afin de le re-dessiner.
  \item Pour rendre l'application plus attractive et moins limité aux amateur de cyclisme, elle pourrait adopter le comportement d'un réseau
  social, avec plus d'interaction entre les utilisateurs (fil d'actualité, messages privés) et plus de sports utilisant ce principe (course à pied, moto).
\end{enumerate}