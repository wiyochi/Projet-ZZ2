\graphicspath{{Others/}}

\section{Déroulement du projet}
\subsection{Organisation théorique du travail}
\subsubsection{Répartition des tâches et prévision de l'emploi du temps}
Le projet fut dès le départ pensé dans le but d'être simple à séparer sous formes de modules, permettant de travailler en parallèle sur plusieurs fonctionnalités.
De plus nous comme nous avions nous même proposé le sujet, il fut assez difficile de prévoir une charge de travail associée à chaque module. Nous avons donc estimé 
de manière très grossière le temps de travail par module. Pour être sûr de pouvoir ajuster le déroulement du projet, nous avons prévus des modules de durées différentes permettant ainsi
de choisir un module en fonction du temps restant, c'est pour cette raison que la durée estimée est supérieure au 60H par personne que nous sommes censés faire.
\par
Nous avions prévu, lors de notre premier rendez-vous avec notre tuteur, de travailler 4H par semaine de cours et de ne pas travailler les semaines de vacances.
Nous avons sommes donc parvenu à finaliser l'emploi du temps suivant, qui n'avait pas pour but d'être suivis à la lettre.
\vfill
\begin{figure}[!h]
    \begin{center}
        \includegraphics[height=6cm]{test-1.png}
        \caption{Première partie du diagramme de Gantt prévisionnel}
    \end{center}
\end{figure}
\subsubsection{Explications sur les tâches}
Les deux premiers objectifs fixés étaient assez simples, leur but étaient de nous laisser le temps d'être à l'aise avec les technologies choisies.
Nous devions prévoir la base de données, c'est à dire la concevoir et la mettre en place sur la machine virtuelle. En parallèle de celà, nous devions réussir à récupérer les coordonnées GPS du téléphone, et réussir à les afficher.
\par
Les objectifs suivant étaient d'enrichir l'expérience utilisateur en améliorant l'interface. Nous voulions en premier permettre la gestion d'un compte utilisateur depuis l'application, ce qui implique un écran de connexion, un écran de création de compte ainsi qu'un écran de gestion de compte.
Dans un second point (développé en parallèle) nous devions enrichir l'interface fonctionnelle de l'application, c'est à dire insérer une interface contenant une carte sur laquelle notre trajet serait affichable (celà sous entend de stocker les coordonnées aquises).
\par
Nous avions prévus de faire évoluer l'application en rajoutant du contenu. Il aurait fallu ajouter des statistiques plus complètes sur les trajets effectués, comme par exemple$\ :$ le dénivelé, la météo ou bien une estimation des calories dépensées.
Il fallait également introduire une gestion des utilisateurs plus développée, qui permettrait de gérer plus finement les droits d'accès. On aurait ainsi pu dire qu'un autre utilisateur avait participé à un trajet, ou bien qu'il avait le droit d'en modifier le contenu.
De la même manière, on aurait pû créer des groupes d'utilisateurs pour un club par exemple. Dans ces groupes tout le monde aurait accès en lecture uniquement sauf les administrateurs. Ainsi un club sportif aurait pû utiliser l'application pour organiser des séances de randonnées.
\par
Un des derniers points à mettre en place était l'affichage des statistiques précédemments acquises sous la forme de graphique où l'on aurait pu choisir l'échelle, et les trajets qui rentraient en compte.
Le dernier point était radicalement plus difficile à traiter, nous voulions finir le développement de l'application en la faisant se rapprocher d'un réseau social. On aurait alors pu avoir des amis, un fils d'actualité contenant les trajets (publiques) de nos amis. On aurait aussi pu partager
nos trajets via des liens webs, qui auraient été ouvrables uniquement par notre application.
\par
Enfin il y avait la dernière tâche qui semble évidente qui était la rédaction du rapport. Nous avions prévu de prendre des notes au fur et à mesure du développement du projet pour parvenir à rédiger le rapport plus efficacement.
\vfill
\begin{figure}[!h]
    \begin{center}
        \includegraphics[height=6cm]{test-2.png}
        \caption{Seconde partie du diagramme de Gantt prévisionnel}
    \end{center}
\end{figure}






\subsection{Organisation réelle du travail}
\subsubsection{Répartion des tâches et emploi du temps}
Au lieu de débuter le projet comme prévu$\ :\ $chacun sur un module, nous avons préféré faire quelques séances de travail en commun afin de découvrir ensemble l'environnement android, et de nous mettre entièrement d'accord sur la suite.
\par
Nous avons ensuite séparer le travail, un s'est chargé du développement android et l'autre de la base de données. Développer sous android implique nécessairement d'en étudier plus le fonctionnement. D'autant plus que nous voulions utiliser Kotlin qui est un langage que nous ne connaissions pas du tout.
Pour la base de donnée le plus difficile était de d'en faire l'installation et la configuration.
\par
Les partiels qui ont suivis nous ont posés beaucoup de problèmes, nous avons été assez surpris par la charge de travail à fournir sur deux semaines. En plus nous avons eu plusieurs TPs important à rendre la semaine suivante.
Nous nous sommes replongés dans le travail la semaine précédant les vacances. Nous avons eu le temps d'intégrer l'interface finale de l'application avec le menu sur le coté gauche. Et nous avons commencé le développement du serveur qui allait intéragir avec la base de données.
\par
Ensuite il y a eu les vacances et un problème abordé dans la partie "Problèmes rencontrés". Puis nous avons commencés à développer la communication entre le serveur et l'application ainsi que l'affichage d'un trajet sur une carte. Un second problème est survenu ce qui nous a imposé de mettre en pause le développement du serveur, pour finalement ne jamais le reprendre.
Nous avons donc commencé le rapport. L'application s'est étoffée pour fournir l'enregistrement de trajets en local, ainsi que leur sauvegarde (sur le téléphone) sous la forme d'un historique (accessible depuis l'application).
Ces dernières fonctionnalités étant fournies par une librairie de gestion de nos structures de données.
\par
Enfin nous avons dû arrêter le développement de l'application pour nous concentrer sur l'écriture du rapport de projet.
\vfill
\begin{figure}[!h]
    \begin{center}
        \includegraphics[height=9cm]{reel-1.png}
        \caption{Première partie du diagramme de Gantt réel}
    \end{center}
\end{figure}
\newpage
\subsubsection{Méthodologie de travail}
Nous avons essayer autant que possible de nous organiser à l'avance sur le travail. C'est à dire qu'en fin de chaque séance nous avons essayé de fixer un objectif à réaliser à la séance suivante.
Pendant le deuxième semestre l'emploi du temps que nous avions a fait que nous avons travaillé plus souvent séparément. Nous parlions alors régulièrement de notre avancement respectif en se donnant une date limite pour chaque tâche.
\par
Le développement de la librairie de gestion des trajets, nous avons naturellement mis en place une méthode de travail agile, sous la forme d'itérations. Ainsi lorsqu'une fonctionnalité de l'application avait besoin d'un accès à la structure de données, cette partie de la librairie était développée.
Cela nous a permis de travailler efficacement, en mettant le doigt rapidement sur ce qui ne convenait pas à l'application.
\par
Pour ce qui est de la communication avec notre tuteur, nous devions envoyé des mails toutes les semaines pour faire suivre l'avancement du projet. Dans la pratique, nous avons été moins assidus. Nous avons cependant gardé le contact au fur et à mesure des évolutions, ainsi que lorsque nous avions des difficultés.
\vfill
\begin{figure}[!h]
    \begin{center}
        \includegraphics[height=9cm]{reel-2.png}
        \caption{Seconde partie du diagramme de Gantt réel}
    \end{center}
\end{figure}
\newpage




\subsection{Problèmes rencontrés}

