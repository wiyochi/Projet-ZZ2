\section{Choix de conception}

\subsection{Base de données}
Nous avons tout d'abord dû choisir quelle technologie nous allions utiliser pour notre base de données.
Nous en connaissions un nombre suffisant pour ne pas en chercher d'autres. Notre choix s'est finalement porté sur MySQL car il répondait à nos besoins, et sa mise en place était plus accessible.
\par
\begin{center}
    Technologies de bases de données connues
    \par
    \begin{tabular}{|l|c|c|c|c|}
        \hline
        Nom & Niveau de connaissance & Facilité d'intégration \\
        \hline
        PostGreSQL & Inconnu & Oui\\
        \hline
        MySQL & Bon & Oui \\
        \hline
        Oracle Database & Bon & Oui\\
        \hline
        sqlite3 & Bon & Non\\
        \hline
    \end{tabular}
    \addcontentsline{lot}{table}{Technologies de bases de données connues}
\end{center}


\subsection{Application}
Pour développer la partie client de l'application, le choix s'est porté naturellement sur une application Android puisque nous possèdons des
smartphones sous Android 9 et Android 7. 
\par
Nous avons donc utilisé Android Studio, un environnement de développement intégré conçu pour générer
des applications Androids. Ce logiciel utilise le langage XML pour la partie "statique", visuelle, et nous laisse le choix entre le langage
Kotlin et le langage Java pour la partie exécution de code.
