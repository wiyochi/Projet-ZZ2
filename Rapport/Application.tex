\documentclass{article}

\usepackage[utf8]{inputenc}
\usepackage[T1]{fontenc}
\usepackage[francais]{babel}
\usepackage{xcolor}
\usepackage{listings}
\usepackage{mathptmx}
\usepackage{anyfontsize}
\usepackage{t1enc}
\usepackage[top=2cm, bottom=2cm, left=2cm, right=2cm]{geometry}
\usepackage{titlesec}
\usepackage{titling}
\usepackage{graphicx}
\usepackage{csquotes}
\usepackage[colorlinks = true,
            linkcolor = black,
            urlcolor  = black,
            citecolor = black,
            anchorcolor = black]{hyperref}

\newcommand{\changeurlcolor}[1]{\hypersetup{urlcolor=#1}}

\renewcommand\maketitlehooka{\null\mbox{}\vfill}
\renewcommand\maketitlehookd{\vfill\null}

\definecolor{codegreen}{rgb}{0,0.6,0}
\definecolor{codegray}{rgb}{0.5,0.5,0.5}
\definecolor{codepurple}{rgb}{0.58,0,0.82}
\definecolor{backcolour}{rgb}{0.95,0.95,0.92}
\definecolor{codekeywords}{rgb}{0.1,0.53,0.92}

\lstdefinestyle{c++}{
    backgroundcolor=\color{backcolour},   
    commentstyle=\color{codegreen},
    keywordstyle=\color{codekeywords},
    numberstyle=\tiny\color{codegray},
    stringstyle=\color{codepurple},
    basicstyle=\ttfamily\footnotesize,
    breakatwhitespace=false,         
    breaklines=true,                 
    captionpos=b,                    
    keepspaces=true,                 
    numbers=left,                    
    numbersep=5pt,                  
    showspaces=false,                
    showstringspaces=false,
    showtabs=false,                  
    tabsize=2,
    texcl=false,
    inputencoding=utf8x,
    extendedchars=true,
    literate=
  {á}{{\'a}}1 {é}{{\'e}}1 {í}{{\'i}}1 {ó}{{\'o}}1 {ú}{{\'u}}1
  {Á}{{\'A}}1 {É}{{\'E}}1 {Í}{{\'I}}1 {Ó}{{\'O}}1 {Ú}{{\'U}}1
  {à}{{\`a}}1 {è}{{\`e}}1 {ì}{{\`i}}1 {ò}{{\`o}}1 {ù}{{\`u}}1
  {À}{{\`A}}1 {È}{{\'E}}1 {Ì}{{\`I}}1 {Ò}{{\`O}}1 {Ù}{{\`U}}1
  {ä}{{\"a}}1 {ë}{{\"e}}1 {ï}{{\"i}}1 {ö}{{\"o}}1 {ü}{{\"u}}1
  {Ä}{{\"A}}1 {Ë}{{\"E}}1 {Ï}{{\"I}}1 {Ö}{{\"O}}1 {Ü}{{\"U}}1
  {â}{{\^a}}1 {ê}{{\^e}}1 {î}{{\^i}}1 {ô}{{\^o}}1 {û}{{\^u}}1
  {Â}{{\^A}}1 {Ê}{{\^E}}1 {Î}{{\^I}}1 {Ô}{{\^O}}1 {Û}{{\^U}}1
  {œ}{{\oe}}1 {Œ}{{\OE}}1 {æ}{{\ae}}1 {Æ}{{\AE}}1 {ß}{{\ss}}1
  {ç}{{\c c}}1 {Ç}{{\c C}}1 {ø}{{\o}}1 {å}{{\r a}}1 {Å}{{\r A}}1
  {€}{{\EUR}}1 {£}{{\pounds}}1
}
\lstset{style=c++}


\title{Ouai}
\author{Arquillière Mathieu - Zangla Jérémy}
\date{\today}

% \begin{lstlisting}[style=c++]
% \end{lstlisting}
% \changeurlcolor{blue}\href{run:../Documentation/latex/refman.pdf}{Lien vers la documentation Doxygen}

\begin{document}

\begin{titlepage}
  \maketitle
\end{titlepage}

\tableofcontents
\listoffigures
\newpage

\section{La mission}
L'objectif de la réalisation du côté client est d'obtenir une application android

\section{Prise en main de l'application}
\subsection{Organisation de l'application}
L'application se rapproche beaucoup d'une application android standard. Elle possède une page d'accueil vide à notre stade de développement mais
elle est censée contenir à terme les derniers trajets effectués et partagés par nos amis dans l'application.

Un menu de navigation est accessible depuis le bouton en haut à gauche de l'écran ou en faisant glisser le doigt de gauche à droite de l'écran.
Ce menu permet d'accéder aux différentes fonctionnalités de logiciel. Ce menu devait également afficher notre nom d'utilisateur et
possiblement un icône pour nous représenter.

\subsection{Onglet "Nouveau Trajet"}
Cet onglet est la partie principale de l'application. Il contient une carte dynamique (google map) et un bouton. Lorsqu'on clique sur ce
bouton, on lance alors la création d'un nouveau trajet. Cela se remarque grâce au bouton qui a changé de texte et de fonction, il permet
alors d'indiquer la fin du trajet, et de la carte qui affiche maintenant un point nous représentant sur la carte. Dès lors, lorsqu'on bouge
le gps le détecte et transmet un nouveau point à la carte. Notre position change donc et la suite de ces points affiche un chemin visible
sur la carte.

Une fois le trajet voulu réalisé, on clique sur le bouton en bas de l'écran pour le terminer. Ceci a pour effet de faire apparaître une
petite fenêtre en superposition de la carte. Ce "pop-up" contient une zone de texte et un bouton et nous permet de rentrer un nom pour
le trajet que l'on vient d'effectuer. Une fois ce nom rentré, le trajet s'enregistre sur le téléphone et on peut de nouveau créer un
autre trajet.

\subsection{Onglet "Historique"}
Cet onglet contient l'ensemble des trajets effectués et enregistrés. Ils s'affichent du plus récent au plus vieux. Chaque trajet est représenté
par une "card", un conteneur composé du nom du trajet, de sa date de création, de sa durée et d'une capture d'écran de la carte prise au moment
de sa création. Si il y a trop de trajets et qu'ils ne rentrent pas tous dans l'écran, on peut les faire défiler grâce à une barre de défilement
en faisant glisser son doigt de bas en haut. Chaque "card" est cliquable et amène vers une page dédié au trajet.

\subsection{Page "Trajet"}
Une page de trajet s'affiche lorsqu'on clique sur l'un dans l'historique. Cette page détaille le trajet avec son nom, sa date de création et
sa durée mais créer aussi une nouvelle carte (google map) sur laquelle est retracée le chemin effectué lors de la création du trajet.

\section{Organisation}
\subsection{Arborescence}

\newpage
\appendix
\section{Manuel d'utilisation}

\end{document}