\section{Introduction}
L'outil numérique a envahi progressivement toutes les sphères de notre vie quotidienne. Cet outil nous semble de
plus en plus indispensable et c'est tout naturellement que, pratiquant une activité de vélo tout terrain en forêt,
nous avons souhaité developper une application en lien.

En effet, en explorant différents chemins on souhaite pouvoir s'en souvenir et les réemprunter. Le projet serait de
proposer la possibilité de repérer, baliser et répertorier ces chemins, puis de les partager avec une communauté
amateur de cyclisme.

Notre problématique est donc de répondre à ce besoin en prenant en compte les restrictions liées au sport. La
solution doit être portable, utilisable pour un cycliste. Elle doit utiliser la géolocalisation et une base de données.

Le smartphone est un support qui convient parfaitement, muni le plus souvent d'un GPS, d'une connexion internet
et d'une taille permettant d'être transporté en vélo.

Notre projet se déroule dans le cadre d'un projet de deuxième année d'école d'ingénieurs en informatique à ISIMA. C'est
dans ce contexte que nous proposons une solution.

Nous verrons tout d'abord dans ce rapport notre démarche pour analyser la problèmatique, puis ce que nous avons mis
en oeuvre pour développer une solution et enfin quel est le résultat produit et ses possibles améliorations.